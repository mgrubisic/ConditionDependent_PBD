\chapter{Study Gap}

\section{Research Gap}

As was stated previously, there is considerable discussion in the literature over the choice of viscous damping models for NLTHA of structures. The very fact that such a discussion is taking place, and gaining momentum, illustrates its broad impacts for performance-based design. As the engineering community continues to advance Performance Based Design, accurate estimates of deformation capacities and demands become essential. Such calculations were irrelevant in the force-based era, but are now vital.  

Some researchers have agreed on the need to conduct a deep parametric study regarding the impact of various damping models on the nonlinear response of structures. Furthermore, the research that has been done mainly involves the consequences of the selection of damping models for the inelastic response of buildings where non-structural components contribute to damping. In the cases of bridges, where the systems approach are based on a ‘bare-frame’, limited studies have been conducted. 

This research will develop a sensitivity analysis based upon NLTHA that considers numerous bridge typologies and damping models. It is expected that this research will help guide engineers on the consequences of the use/misuse of different damping models in non-linear analysis. In addition, it will provide a framework where practitioners might consider using Direct Displacement-Based Design (DDBD) as an alternative tool not only to design common highway bridges but also to verify procedures that are more complex.

\section{General Objectives}
Guide engineers on the consequences of the use/misuse of different damping models in the non-linear analysis of bridges.

\section{Specific Objectives}