\chapter{LITERATURE REVIEW}
\label{chap-two}
In this chapter the available knowledge on the different topics that are available in the literature are summarized. First a review on the different definitions of commutative damage is presented then the main idea for this research are established and the required components, then the different elements that form part of this study are presented and then a general concept is established and presented in Chapter 3.

\section{Cumulative Damage}

Cumulative Damage in structures have been tried to be established for structures to identify the state of a structure 

The best-known and most widely used of all the cumulative damage index is that of Park and Ang (1985). This consists of a simple linear combination of normalized deformation and energy absorption: 

The first term here is a simple, pseudo-static displacement measure. It takes no account of cumulative damage, which is accounted for solely by the energy term. The advantages of this model are its simplicity, and the fact that it has been calibrated against a significant amount of observed seismic damage, included some instances of shear and bond failures. Park, Ang and Wen (1985) suggested D = 0.4 as a threshold value between repairable and irreparable damage, while the same authors in 1987 suggested the following more detailed classification: 

\subsection{Damage Index}
\subsection{Fragility Curves}
